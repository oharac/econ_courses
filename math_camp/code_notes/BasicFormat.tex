%Document Class
	\documentclass{article}
	
%Packages
	\usepackage{amsmath}
	\usepackage{amssymb}
	\usepackage[margin=0.5 in]{geometry}
	\usepackage{mathtools}
	\usepackage{tikz}
		\usetikzlibrary{arrows}
	
%Other Document Settings
	\setlength\parindent{0pt}

%Document
\begin{document}
James Banovetz\\
Math Camp 2017\\
9/28/2017

\begin{enumerate}
%%%%%%%%%%%%%
%PROBLEM ONE%
%%%%%%%%%%%%%
\item\textbf{Let $x$ and $y$ be integers. Prove the following proposition: If $x$ and $y$ are even,
	then $xy$ is even.}

	\underline{\smash{To show}}: there exists an integer $h$ such that $xy=2h$.\\
	\underline{\smash{Proof}}:\\[-18pt]
	\begin{align*}
		&\text{Let $x$ and $y$ be even integers}
				&\text{(by hypothesis)}\\
		&\implies\Big(\exists k\in\mathbb{Z}\ni x=2k\Big)\wedge\Big(\exists j\in\mathbb{Z}\ni y=2j\Big)
				&\text{(by def. of even)}\\
		&\implies xy = (2j)(2k)																		
				&\text{(multiplying)}\\
		&\implies xy = 2(2jk)
				&\text{(by associativity)}\\
		&2jk\in\mathbb{Z}
				&\text{(by closure)}\\
		&\text{Let }h=2jk
				&\text{(defining an integer $h$)}\\
		&\implies xy = 2h
				&\text{(substituting for $2jk$)}\\
		&\implies xy\text{ is even}	
				&\text{(by def. of even)}\\
		&&\blacksquare
	\end{align*}



%%%%%%%%%%%%%
%PROBLEM TWO%
%%%%%%%%%%%%%
\item\textbf{Solve the following constrained utility-maximization problem.
		\[\max_{x_1,x_2}\alpha \ln(x_1) + x_2		\hspace{5mm} \text{s.t.} \hspace{5mm}
			M \geq p_1x_1 + p_2x_2		\hspace{5mm} \text{and} \hspace{5mm}
			x_1 \geq 0, x_2 \geq 0
		\]
	where $\alpha>0$, $M>0$, $p_1>0$, and $p_2>0$.}

	The Lagrangian associated with this problem is:
		\[\mathcal{L}(x_1,x_2,\lambda)=\alpha\ln(x_1)+x_2+\lambda\big[M-p_1x_1-p_2x_2\big]\]

	The Kuhn-Tucker first-order conditions are:
	\begin{align*}
		\mathcal{L}_1 & = \frac{\alpha}{x_1} - p_1\lambda \leq 0
			&x_1 & \geq 0
			&x_1\mathcal{L}_1 & = 0
				\\
		\mathcal{L}_2 & = 1 -p_2\lambda \leq 0						
			&x_2 & \geq 0
			&x_2\mathcal{L}_2 & = 0
				\\[5pt]
		\mathcal{L}_\lambda & = M - p_1x_2 - p_2x_2 \geq 0
			&\lambda & \geq 0
			&\lambda\mathcal{L}_\lambda & = 0
	\end{align*}

	We don't have to go through all of the possible combinations of binding/non-binding constraints
	if we use a little economic intution:
	\begin{itemize}
	\item The utility function is strictly increasing, so $M=p_1x_1+p_2x_2$
	\item $\ln(x_1)\rightarrow-\infty$ as $x\rightarrow 0$, so $x_1>0$
	\end{itemize}

	Thus, there are only two potential solutions. Assuming positive consumption of both goods
	(only the budget constraint binds):
	\begin{align*}
		0 & = \frac{\alpha}{x_1}-p_1\lambda
				&\text{(from the first FOC)}\\
		0 & = 1-p_2\lambda
				&\text{(from the second FOC)}\\[5pt]
		M & = p_1x_1+p_2x_2
				&\text{(from the third FOC)}
	\intertext{Manipulating the first two FOCs:}
		\frac{1}{p_2} & = \frac{\alpha}{p_1x_1}
				&\text{(combining the 1st two FOCs)}\\
		\Aboxed{x_1^* & = \alpha\left(\frac{p_2}{p_1}\right)}
				&\text{(solving for $x_1$)}
	\end{align*}
	(etc., etc.)



%%%%%%%%%%%%%%%
%PROBLEM THREE%
%%%%%%%%%%%%%%%
\item\textbf{Let $Y$ be a continuous random variable with PDF
	\[f_{Y}(y)=\begin{cases}
			(3/2)y^2+y&\text{if }0\leq y\leq 1 \\[5pt]
			0 &\text{else}
			\end{cases}\]
	Find the mean of $Y$.}
	
	\begin{align*}
		\mathbb{E}[Y]&=\int_0^1y\left[\frac{3}{2}y^2+y\right]\;dy
				&\text{(by def. of expected value)}\\[4pt]
		&=\int_0^1\frac{3}{2}y^3+y^2
				&\text{(simplifying)}\\[4pt]
		& = \left[\frac{3}{8}y^4+\frac{1}{3}y^3\right]_0^1
				&\text{(taking the integral)}\\[4pt]
		\mathbb{E}[Y]& = \frac{17}{24}
				&\text{(evaluating)}
	\end{align*}



%%%%%%%%%%%%%%%
%EXAMPLE GRAPH%
%%%%%%%%%%%%%%%
\item\textbf{Graph a generic function using the ``tikz" package in LaTeX:}
	\begin{center}
		\begin{tikzpicture}
		[shorten >=-2.5pt, shorten <= -2.5pt]
		%AXES
			\draw[very thick, <->] (0,5) node[left]{$y$} -- (0,0) -- (5,0) node[below]{$x$};
		%FUNCTION
			\draw[*-o](0,1)--(1,1);
			\draw[*-o](1,2)--(2,2);
			\draw[o-*](2,3)--(3,3);
			\draw[->](3,3)--(4.5,4.5);
		\end{tikzpicture}
	\end{center}



\end{enumerate}
\end{document}